%%%%%
%
% 14-OCT-2014 Oliver Bach (bach@decfreak.de) finxit
% Template file for LaTeX typesetting with documentclass article
%
%%%%%

\documentclass{article}


\usepackage{ngerman} % Fuer neue deutsche Rechtschreibung

\usepackage{graphicx} % Fuer Grafiken/Bilder

\usepackage{multicol} % Fuer Mehrspaltigen Textsatz

\usepackage{amsmath}




\begin{document}

	\date{13-OCT-2014}
	\title{Eine Einf"uhrung in LaTeX}
	\author{Oliver Bach}

\maketitle

\tableofcontents

	\newpage

\listoftables

\listoffigures

	\newpage

Inhalt

\section{Wald}

	Ein bisschen Inhalt zu Wald

 \subsection{Tannenwald}

	Wir schreiben hier mal irgendwelchen Text.
	Und ich fange hier in einer neuen Zeile an. \\
	Und wieder in einer neuen Zeile, ohne dass das
	irgendwelche Auswirkungen hat.

\newpage

 \subsection{Laubwald}

	Ganz, ganz viel Flie"stext schreiben.

	\begin{itemize}
		\item Hier ist ein Punkt, der ganz viel Text hat und extra lang wird, und dann "uber die n"achste Zeile hinausgeht
		 \subitem - Hier ist ein neuer Punkt
		 \subitem Hier noch etwas
	\end{itemize}


	\begin{table}

		\begin{tabular}[h]{l|c|r}
	 	\hline
			Erste Spalte &	Zweite Spalte	& Dritte Spalte	\\
			links		& mitte		& rechts	\\
		\hline
		\end{tabular}

	\caption{Meine erste Tabelle}
	\label{erstetabelle}

	\end{table}


 \subsubsection{Bl"atter}

	\begin{enumerate}
		\item Hier ist ein Punkt, der ganz viel Text hat und extra lang wird, und dann "uber die n"achste Zeile hinausgeht
		 \subitem Hier ist ein neuer Punkt
		 \subitem Hier noch etwas
		\item Hier ein gro"ser Punkt
	\end{enumerate}
 
 \subsection{Nadeln}

	Diese Tabelle hat kein label und auch keine caption
	
	\begin{tabular}[h]{l|c|r}
 		\hline
		Erste Spalte &	Zweite Spalte	& Dritte Spalte	\\
		links		& mitte		& rechts	\\
		\hline
	\end{tabular}

 \subsection{Mischwald}

 	Diese Ergebnisse finden Sie in Tabelle \ref{erstetabelle} auf Seite \pageref{erstetabelle}

 \subsubsection{"Ubergang}

	Ich will aber Ma"snahmen treffen.

\section{Hof}

\section{Wiese}

	Ein bisschen Inhalt zu Wiese.

\newpage


 \subsection{Bild} \label{sec:bilder}


	\begin{figure}[h]
	
		\begin{center}
	
		\includegraphics[width=0.25\textwidth]{Images/lolcat-sample.jpg}
		\caption{Mein Bild von einer s"u"sen Katze}
		\label{fig:katze}
	
		\end{center}

	\end{figure}


	Das Ergebnis ist in Abbildung \ref{fig:katze} zu sehen.


	\begin{figure}[h]
	
		\begin{center}
	
		\includegraphics[width=0.25\textwidth]{Images/lolcat-sample.jpg}
		\caption{Mein Bild von einer s"u"sen Katze zum zweiten Mal}
		\label{fig:katze2}
	
		\end{center}

	\end{figure}

	Das Finden Sie in Abschnitt \ref{sec:bilder}



	Text,d er mehrspaltig dargestellt wird:

 	\begin{multicols}{4}
	[
	\section{First Section}
	All human things are subject to decay. And when fate summons, Monarchs must obey. \\
	--- Dies wird noch in einer Spalte dargestellt.
 	]
	Hello, here is some text without a meaning.  This text should show what 
	a printed text will look like at this place.
	If you read this text, you will get no information.  Really?  Is there 
	no information?  Is there...
	Hier schreibe ich noch weiteren Inhalt dazu.
	Und noch mehr Inhalt dazu.
	\end{multicols}
   
\newpage

	Ein paar Formel: 
	$\sum^{9}_{i=5}{4*i}$
	
	oder ein Bruch: $\frac{5}{6}$

	
   \vspace*{2cm}

  	Hier eine gro"se Formel, die durch die Verwendung von zwei \${}\${} gr"o"ser gesetzt wird,
	also das obige Beispiel mit nur einem \${}:

   	$$ \frac{67}{\sum^{9}_{i=5}{4*i}} $$





\newpage

	\textbf{fett} \textit{kursiv} \underline{unterstrichen} \\

	\textbf{ \textit{ \underline {hier steht etwas unterstrichenes kursives und fettes } }  }


	Sonderzeichen, die von \LaTeX belegt sind:
	\&
	
	\$
	
	\%
	
	$\backslash $

	Die verschiedenen Bindestriche, und das Minus:
	- \\
	-- \\
	--- \\
	$-$

	
	
	W\"alder, W\"elder

	``Hier etwas in Anf"uhrungszeichen''\footnote{irgendwas dazu}

	\glqq Hier ein Zitat in Deutsch \glq{}mit einfachem Anf"uhrungszeichen\grq{} \grqq

	Diesen Gedanken finden sie auch bei\footnote{ \cite[S. 13]{AG14-2014} }

\newpage


\begin{thebibliography}{Gloger-hihihi2013}

	\bibitem[AG14-2014]{AG14-2014}
		Gloger, Stefan: \textit{irgendwas wichtiges}.
		Pieper-Verlag, M"unchen, 2014, ISBN 978-47839-92382-X.
	
	\bibitem[Gloger-hihihi2013]{gloger}
		Gloger, irgendwer $\ldots$

\end{thebibliography}


\end{document}
