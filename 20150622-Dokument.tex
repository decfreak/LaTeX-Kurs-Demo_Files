%%%%%
%
% 22-JUN-2015 Oliver Bach (bach@decfreak.de) finxit
% Template file for LaTeX typesetting with documentclass article
%
%%%%%

%% An oberster Stelle wird die Dokumentenklasse definiert
\documentclass{article}

%% Anschlieszend folgen die Pakete, welche im Dokument verwendet werden sollen
\usepackage{ngerman}
\usepackage{graphicx}
\usepackage{url}
\usepackage{verbatim}
\usepackage[ngerman]{babel}
\usepackage{todonotes}
%\usepackage[disable]{todonotes} %% Diese Zeile schaltet alle noch vorhandenen ToDos im Dokument ab


%% Hier beginnt der eigentliche Inhalt des Dokuments
\begin{document}

\pagenumbering{roman} %% Seitennummerierung mit Roemischen Seitennummern

\tableofcontents %% Inhaltsverzeichnis

\newpage %% Eine neue Seite beginnen

\listoftodos %% 


\newpage

\pagenumbering{arabic} %% Wechsel zu arabischen Seitennummern - Es wird wieder bei Seite 1 begonnen zu zaehlen

%% Ein Abschnitt der Stufe "Section"
 \section{Abschnitt}

 Blub

%% Ein neuer Abschnitt der Stufe "Section"
 \section{Noch ein Abschnitt}
 
 Hallo \todo{Hier gehoert noch was hin} weiterschreiben %% Ein ToDo mitten in der Zeile

%% Eine Aufzaehlung, welche durchnummeriert ist
\begin{enumerate}
	\item{ Erster Punkt}
	\subitem Unteritem
	\begin{itemize}
		\item[*] bla
		\item[0] blub
	\end{itemize}
	\item Zweiter
	\item Blub
\end{enumerate}


%% Die gleiche Aufzaehlung, nur mit Aufzaehlungspunkten
\begin{itemize}
	\item{ Erster Punkt}
	\subitem Unteritem
	\begin{itemize}
		\item[*] bla
		\item[0] blub
	\end{itemize}
	\item Zweiter
	\item Blub
\end{itemize}


 viele leerzeilen \todo[inline]{Hier ein Breites ToDo}
 keine leerzeile

 eine leerzeile \newline
 oder die Kurzform \\
 bisscen text

%% Ein Unterabschnitt der Stufe "SubSection"
  \subsection{Unterabschnitt}
	Bisschen Text im Unterabschnitt

  \subsection{Noch ein Unterabschnitt}

  \subsection*{Dritter Unterabschnitt}
  	Dieser wird nicht durchnummeriert.
	Des\-oxyribo\-nuklein\-s"aure.
	Stra"se

 \subsection{Vierter Unterabschnitt}
 	\label{sec:Maier}

	Noch ein unterabschnitt, der nummeriert wird.

	Deutschedonaudampfschifffahrtsgesellschaft.
	Jopsfienschen

\section{Tabelle}

%% Der Beginn einer Table-Umgebung

\begin{table}
	\label{tab:hoffmann} %% Ein Label zum Referenzieren der Tabelle

%% Die eigentliche Tabelle innerhalb der Table-Umgebung
\begin{tabular}{lcrlll}
	%\hline
	links 	& \% mitte	& rechts	& & & \\	
	links	& mittig	& \begin{tabular}{p{0.25\textwidth}}
					hallo
				\end{tabular} & & & \\
	%\hline

\end{tabular}

	\caption{Meine Tabelle} %% Name der Tabelle
\end{table}

\section{Ein Bild}
	\label{sec:Hoffmann}

%% Eine Figure-Umgebung fuer Bilder
	\begin{figure}
		\includegraphics[height=0.35\textheight]{Images/lolcat-sample.jpg} \\
		Quelle: Bach(2015)
		\caption{Unterschrift des Bildes}
		\label{fig:bild1}
	\end{figure}



\section{Formeln}

 Jetzt kommen mitten im Text $ \sum_{i=1}^{100} 1+i $ eine Formel %% Eine Formel in der Textzeile
 \footnote{Siehe \cite[S. 307]{ulmann-2015} } %% Verwendung einer Fusznote

 Die Formel ist gegeben als:

 hier noch etwas in der Zeile
 \begin{displaymath} %% Formeln grosz und zentriert darstellen
 \sum_{i=1}^{100} 1+i 
 \end{displaymath} in der zeile weiterschreibe

 	Wird berechnet aus $1+Kostenfaktor*Einkaufspreis$

\newpage


Dieser Sachverhalt ist zu sehen in Abbildung \ref{fig:bild1} auf Seite \pageref{fig:bild1}. %% Referenzen auf Abbldung und deren entsprechende Seite


\glqq{}bla \glq{}Ein Zitat im Zitat\grq{} bla\grqq{} %% Deutsche einfache und doppelte Anfuehrungszeichen

%% Erneuer Wechsel zu romieschen Seitennummern
\pagenumbering{roman}
\setcounter{page}{2} %% Setzen des Seitenzaehlers, damit entsprechend weitergezaehlt wird.

%% Abbildungsverzeichnis
\listoffigures

%% Tabellenzeichnis
\listoftables

%% Literaturverzeichnis
\begin{thebibliography}{Ulmann(2015)}
	\bibitem[Ulmann-2015]{ulmann-2015}  %% Ein Eintrag im Literaturverzeichnis mit Kurzname und Referenz
		Ulmann, Bernd: \textit{Sage}.
			ISB: 1-2345-6789, 2014.
	\bibitem[Maier-1998]{maier-1998}
		Maier, Felix.
		Webseite: \url{http://www.google.de}, Letzter Aufruf: 22-JUN-2015, 20:36.
\end{thebibliography}

%% Ende des Inhalts des Dokuments
\end{document}
