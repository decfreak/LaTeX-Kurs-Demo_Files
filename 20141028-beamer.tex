%%%%%
%
% 28-OCT-2014 Oliver Bach (bach@decfreak.de) finxit
% Template file for beamer presentations with LaTeX
%
%%%%%

\documentclass{beamer}

\usepackage{marvosym}
\usepackage{ngerman}
\usepackage{graphicx}
\usepackage{beamerthemesplit}

%\usetheme{Rochester}		% Different Themes
\usetheme{Darmstadt}

% \useinnertheme{rectangles}
 \useinnertheme{rounded}
 \usecolortheme{rose}

\setbeameroption{show notes}
\setbeamercovered{transparent} %%% Transparenz einschalten

  \title[Kurztitle f"ur Fu"sleiste \hspace{3.5cm}  \insertframenumber]
 	{\emph{Kompletter Titel der Pr"asenation}, \\
		\small mit zweiter Zeile}
 \author[Oliver Bach]{Oliver Bach, B.Sc. \\ (bach\MVAt{}decfreak.de)}
 \date{Version 0.9, Stand: \today}


\begin{document}
 
%%
 \begin{frame}
  \titlepage   		% Titelseite
  
 \end{frame}
%%
\begin{frame}
 \frametitle{"Ubersicht}
 \tableofcontents
\end{frame}
%%
\section{Erster Abschnitt}
 \subsection{}
%%
\begin{frame}
 \frametitle{Eine Folie}

	Ein bisschen Text
\end{frame}

\note
{
Meine Notiz
\begin{itemize}
	\item Bla
	\item blub
\end{itemize}
}
%%
\begin{frame}
 \frametitle{Aufz"ahlung}

 \begin{itemize}[<+->]
 	\item erster Punkt
	\item zweiter Punkt
	\item dritter Punkt
	\item<2> vierter Punkt
 \end{itemize}

\end{frame}
%%
\section{Zweiter Abschnitt}
 \subsection{Bla}
%%
\begin{frame}
 \frametitle{Normaler Text, nacheinander}
 Hier kommt mein Text\\
 \pause
 und dies hier erst als zweites \\
 \pause
 Und mein dritter Text
\end{frame}
%%
\subsection{blub}
%%
\begin{frame}
 \frametitle{Bl"ocke}

 \begin{block}{Name des Blocks}
	Das hier ist ein allgemeiner Einf"uhrungstext
 \end{block}

 \begin{alertblock}{Definition}
	Hier kommt eine wichtige Definition
 \end{alertblock}

 \begin{exampleblock}{Beispiel}
 	Und ein Beispiel dazu

 \end{exampleblock}

 \begin{block}{}
	Ein Block ohne Name
 \end{block}

\end{frame}
%%
\section{Dritte und letzte Section}
%%
\begin{frame}
 \frametitle{Bild}
 	\begin{center}
		\includegraphics[width=0.75\textwidth]{Images/lolcat-sample.jpg}
	\end{center}
\end{frame}
%%
\begin{frame}
 \frametitle{Formel}
	Eine Formel: $\sum_{i=1}^{100} a ^ i$
	\\
	\pause
	Eine gro"se Formel: $$ \sum_{i=1}^{100} a ^ i $$ \\
	\pause
	Von a $\rightarrow$ nach b $\Rightarrow$ und c bzw. $\Leftrightarrow$
\end{frame}
%%
\begin{frame}[fragile]	% option 'fragile' is needed when using verbatim on a slide!
 \frametitle{Code}

 \begin{verbatim}
	Hier steht
	  mein Code
 \end{verbatim}
 
\end{frame}
%%
\begin{frame}[allowframebreaks]	% LaTeX can automatically break the content up in different slides
 \frametitle{Quellen}
	\begin{thebibliography}{Sommerville-2001}
		\bibitem[Vo"s-2012]{Voss-2012}[Vo"s-2012]
			Vo"s, Herbert: \textit{Einf"uhrung in \LaTeX} ---
				unter Ber"ucksichtigung von pdfLaTeX XeLaTeX LuaLaTeX.
			Lehmanns Media Verlag, Berlin, 2012, 1. Auflage, ISBN 978-3-86541-462-5.
		\bibitem[Kloeckl-2002]{Kloeckl-2002}[Kloeckl-2002]
			Kl"ockl, Ingo: \textit{LaTeX Tipps \& Tricks}.
			dpunkt Verlag GmbH, 2. Auflage, 2002, ISBN 3-89864-145-7.
		\bibitem[Liebing-1996]{Liebing-1996}[Liebing-1996]
			Liebing, A.: \textit{Erste Arbeiten mit TeX}.
			Prentice Hall Verlag GmbH, 1996, ISBN 3-8272-9521-1.
		\bibitem[Schultis-1995]{Schultis-1995}[Schultis-1995]
			Schultis, J. Kenneth: \textit{LaTeX-TiPS} -- Praktische Hinweise zur Erstellung technischer Dokumente.
			Prentice Hall Verlag GmbH, 1995, ISBN 3-930436-25
		\bibitem[Vo"s-2012]{Voss-2012}[Vo"s-2012]
			Vo"s, Herbert: \textit{Einf"uhrung in \LaTeX} ---
				unter Ber"ucksichtigung von pdfLaTeX XeLaTeX LuaLaTeX.
			Lehmanns Media Verlag, Berlin, 2012, 1. Auflage, ISBN 978-3-86541-462-5.
		\bibitem[Kloeckl-2002]{Kloeckl-2002}[Kloeckl-2002]
			Kl"ockl, Ingo: \textit{LaTeX Tipps \& Tricks}.
			dpunkt Verlag GmbH, 2. Auflage, 2002, ISBN 3-89864-145-7.
		\bibitem[Liebing-1996]{Liebing-1996}[Liebing-1996]
			Liebing, A.: \textit{Erste Arbeiten mit TeX}.
			Prentice Hall Verlag GmbH, 1996, ISBN 3-8272-9521-1.
		\bibitem[Schultis-1995]{Schultis-1995}[Schultis-1995]
			Schultis, J. Kenneth: \textit{LaTeX-TiPS} -- Praktische Hinweise zur Erstellung technischer Dokumente.
			Prentice Hall Verlag GmbH, 1995, ISBN 3-930436-25
\end{thebibliography}
\end{frame}
%%
\end{document}
